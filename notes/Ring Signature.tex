\documentclass[11pt]{article}
\usepackage[utf8]{inputenc}
\usepackage{mathtools}
\usepackage{amsmath}
\usepackage{amsfonts}
\usepackage{amssymb}
\usepackage[linesnumbered,ruled,vlined]{algorithm2e}
\usepackage[T1]{fontenc}

% Default margins are too wide all the way around. I reset them here
\setlength{\topmargin}{-.5in}
\setlength{\textheight}{9in}
\setlength{\oddsidemargin}{.125in}
\setlength{\textwidth}{6.25in}

\begin{document}
\title{Ring Signature Lecture Notes}
\author{Chi Yin Lee\\
The Hong Kong Polytechnic University\\
Department of Computing}
\renewcommand{\today}{June 20, 2019}

\maketitle
\begin{abstract}
   A ring signature scheme is a group signature scheme with
no group manager to setup a group or revoke a signer. A linkable ring
signature, introduced by Liu, et al. [20], additionally allows anyone to
determine if two ring signatures are signed by the same group member
(a.k.a. they are linked). Ring signature is a type of digital signature that can be performed by any member of a group of users that each have keys. Therefore, a message signed with a ring signature is endorsed by someone in a particular group of people. One of the security properties of a ring signature is that it should be computationally infeasible to determine which of the group members' keys was used to produce the signature. Ring signatures are similar to group signatures but differ in two key ways: first, there is no way to revoke the anonymity of an individual signature, and second, any group of users can be used as a group without additional setup.
\end{abstract}

\section{Proving the Knowledge of Several Discrete Logarithms}
In this section, we describe some three-move interactive HVZK PoK protocols
that we will use as basic building blocks for our event-oriented linkable threshold ring signature scheme. 
These protocols all work in finite cyclic groups of quadratic residues modulo safe prime products. 
For each $i = 1, . . . , n$, let $Ni$ be a safe-prime product and define the group $Gi \doteq QR(Ni)$ such that its order is of length $\ell_i - 2$ for some $\ell_i \in \mathbb{N}$. Also let $g_i, h_i$ be generators of $G_i$ such that their relative discrete logarithms are not known.

Let $1 < \epsilon  \in \mathbb{R}$ be a parameter and let $\mathcal{H} : {0,1}^* \rightarrow \mathbb{Z}_q$ be a strong collision resistant hash function, where $q$ is a $k$-bit prime for some security parameter $k \in \mathbb{N}$. Define $\mathcal{N} \doteq \{1,...,n\}$ and $\Gamma_i \doteq \{-2^{\ell_i}q,...,(2^{\ell_i}q)^\epsilon\}$.

This protocol is a straightforward generalization of the protocol for proving the knowledge of a discrete logarithm over groups of unknown order in [7].
This allows a prover to prove to a verifier the knowledge of $n$ discrete logarithms $x_1,...,x_n \in \mathcal{Z}$ of elements $y_1,...,y_n$ respectively and to the base $g_1,...,g_n$ respectively. Using the notation in [9], the protocol is denoted by:
\begin{equation}
PK\{(\alpha_1,...,\alpha_n) : \bigwedge_{i=1}^{n}{y_i = g_i^{\alpha_i}} \}.
\end{equation}


\section{Classical Ring Signature}
We review the existing constructions of (linkable) ring
signatures. The generic construction introduced by Rivest, Shamir and Tauman [49] in 2001 (RST). This generic construction is based on one-way trapdoor
permutations along with a block cipher. It can be instantiated from the RSA assumption. In 2004, Abe, Ohkubo and Suzuku [1] (AOS) proposed a new generic
construction which allows discrete-log type of keys. This generic construction can
make use of hash-and-sign signature or any three-move sigma-protocol-based signature. It can be instantiated from RSA or discrete-log assumptions. Both of the
RST and AOS constructions are secure in the random oracle model and the signature sizes are linear to the ring size. To achieve the security in standard model,
Bender, Katz and Morselli [12] (BKM) presented a ring signature scheme which
adopts a public-key encryption scheme, a signature scheme and a ZAP protocol
for any language in N P [25]. Even though BKM construction is secure in standard model, the signature size is still linear in the number of group members
and the generic ZAPs are actually quite impractical. Shacham and Waters [?]
then proposed a more efficient linear-size ring signature scheme without random
oracle from bilinear pairing.

To reduce the signature size, Dodis et al. proposed the first ring signature
scheme with constant signature size in 2004 [20]. It relies on accumulator with
one-way domain and is secure in the random oracle model. The first ring signature with sub-linear without random oracle model is due to Chandran, Groth and Sahai [17]. This scheme has signature size $O(\sqrt{\ell})$ where $\ell$ is the number of users in the ring. All of the above sub-linear size constructions are secure in the common reference string model that requires a trusted setup. The first sub-linear ring signatures without relying on a trusted setup is due to Groth and Kohlweiss [30]. It features logarithmic size signature and is secure in the random oracle model.

As noted before, the protocol can be turned into a signature scheme by
replacing the challenge by the hash of the commitment together with the message $M$ to be signed: $c \leftarrow \mathcal{H}((g_1,y_1)||...||(g_n,y_n)||t_1||...||t_n||M)$ where $t_n = g^{s_{n-1}}*y_{n-1}^(c_{n-1})$. In this case, the signature is $(c,s_1,...,s_n)$ and the verification becomes:
\begin{equation}
c \stackrel{?}{=} \mathcal{H}((g_1,y_1)||...||(g_n,y_n)||t_1||...||t_n||M)
\end{equation}

Following [9], we denote this signature scheme by:
\begin{equation}
SPK\{(\alpha_1,...,\alpha_n) : \bigwedge_{i=1}^{n}{y_i = g_i^{\alpha_i}}\}(M).
\end{equation}

With above description, a classic ring signature scheme formed.

With below setup:
\begin{itemize}
\item
Finite Field: $\mathbb{N}$
\item
KeyPair : $\{x_i, y_i\},  y_i = g^{x_i}$
\item
Group : $L=\{y_0,...,y_n\}$
\item
Position: the signer position in Group $L$, $p \in L$
\item
Message: $m$
\item
Ring Size: $n$
\item
Hash Function: $\mathcal{H}(...)$
\end{itemize}

\begin{algorithm}
\DontPrintSemicolon
\KwIn{$\{L, p, m, x_p, g, \mathbb{N}\}$}
\KwOut{$\{C, s_0,...,s_n\}$}

$u \gets \{0, 1\}^\mathbb{N}$\;
$T_p \gets g^u$\;
$ c_p \gets \mathcal{H}(L || m || T_p) $\;
% Ring Operation
\For{$i \gets p+2$ \textbf{to} $p \bmod n$} {
  $s_{i-1} = \{0,1\}^\mathbb{N}$\;
  $T_i = g^{s_{i-1}} . y_{i-1}^{c_{i-1}}$\;
  $c_i = \mathcal{H}(L || m || T_i)$\;
}
$s_p \gets u - c_{p}{x_p}$\;
$C \gets c_0$\;
\Return{$\{C, s_0,...,s_n\}$}\;
\caption{Classic Ring Signature - Sign}
\label{algo:classic-sign}
\end{algorithm}
\begin{algorithm}
\DontPrintSemicolon
\KwIn{$\{L, m, C, s0,...,sn, g, \mathbb{N}\}$}
\KwOut{$true$ or $false$}
\For{$i \gets 0$ \textbf{to} $n$} {
  $T_i = g^{s_i} . y_i^{c_i}$\;
  $c_i = \mathcal{H}(L || m || T_i)$\;
}
\Return{$c_{n} \stackrel{?}{=} C$}\;
\caption{Classic Ring Signature - Verify}
\label{algo:classic-verify}
\end{algorithm}

\subsection{Explanation}
\begin{gather*}
\because y_i = g^{x_i} \\
\therefore T_p = g^{u-c_{p}x_{p}} . y_{p}^{c_p} = g^{u-c_{p}x_{p}} . g^{x_p.c_p} = g^u \\
\therefore c_p = \mathcal{H}(L || m || g^u)
\end{gather*}
Thus, the verifier will recover the origin random number : $g^u$.

\section{Linkable Ring Signature}
Since the first proposal of linkable ring signature [39], we have seen a sequence of work [55, 7, 38, 52] that provides different features. In 2005, Tsang and Wei [55] extends the genric ring siganture introduced by Dodis et al. [20] to a linkable version, which also feature constant signature size and is secure in the random oracle model. Au et al. [7] presented a new security model for linkable ring signatures and a new short linkable ring signature scheme that is secure in this strengthened model. In 2014, Liu et al.
[38] presented the first linkable ring signature scheme achieving unconditional anonymity. Sun et al. [52] proposed a new generic linkable ring signature to construct RingCT 2.0 for Monero. There are also schemes with special properties such as identity-based linkable ring signatures [54, 9] and certificate-based linkable ring signatures [8].
\newline
\begin{equation}
SPK\{(\alpha_1,...,\alpha_n) : \bigwedge_{i=1}^{n}{y_i = g_i^{\alpha_i} \wedge v_i = h_i^{\alpha_i}}\}(M).
\end{equation}
\newline
Given the same setup from Classic Ring Signature , $h$ is a pre-defined variable by the Group.

\begin{algorithm}
\DontPrintSemicolon
\KwIn{$\{L, p, m, x_p, g, h, \mathbb{N}\}$}
\KwOut{$\{C, Y, s_0,...,s_n\}$}
$Y \gets h^{x_p}$\;
$u \gets \{0, 1\}^\mathbb{N}$\;
$T_p \gets g^u$\;
$t_p \gets h^u$\;
$ c_p \gets \mathcal{H}(L || m || T_p || t_p) $\;
% Ring Operation
\For{$i \gets p+2$ \textbf{to} $p \bmod n$} {
  $s_{i-1} = \{0,1\}^\mathbb{N}$\;
  $T_i = g^{s_{i-1}} . y_{i-1}^{c_{i-1}}$\;
  $t_i = h^{s_{i-1}} . Y^{c_{i-1}}$\;
  $c_i = \mathcal{H}(L || m || T_i || t_i)$\;
}
$s_p \gets u - c_{p}{x_p}$\;
$C \gets c_0$\;
\Return{$\{C, Y, s_0,...,s_n\}$}\;
\caption{Linkable Ring Signature - Sign}
\label{algo:linkable-sign}
\end{algorithm}

\begin{algorithm}
\DontPrintSemicolon
\KwIn{$\{L, m, C, Y, s0,...,sn, g, h, \mathbb{N}\}$}
\KwOut{$true$ or $false$}
\For{$i \gets 0$ \textbf{to} $n$} {
  $T_i = g^{s_i} . y_i^{c_i}$\;
  $t_i = h^{s_i} . Y^{c_i}$\;
  $c_i = \mathcal{H}(L || m || T_i || t_i)$\;
}
\Return{$c_{n} \stackrel{?}{=} C$}\;
\caption{Linkable Ring Signature - Verify}
\label{algo:linkable-verify}
\end{algorithm}

\subsection{Ring Explanation}
\begin{gather*}
\because y_i = g^{x_i} \\
\therefore T_p = g^{u-c_{p}x_{p}} . y_{p}^{c_p} = g^{u-c_{p}x_{p}} . g^{x_p.c_p} = g^u \\
\therefore c_p = \mathcal{H}(L || m || g^u)
\end{gather*}
Thus, the verifier will recover the origin random number : $g^u$.

\subsection{Linkable Signer Explanation}
\begin{gather*}
\because Y = h^x_p \\
\therefore t_p = h^{s_p} . Y^c_p = h^{u - c_p x_p} . h^{x_p c_p} = h^u \\
\therefore c_p = \mathcal{H}(L || m || g^u || h^u)
\end{gather*}
Thus, the verifier will recover the origin random number : $h^u$ ; For every signature with the same $h$, $Y$ will be the same as $Y = h^x_p$. Signatures can be linked by checking $\Delta{Y} \stackrel{?}{=} Y$.

\section{Threshold Linkable Ring Signature}
Threshold cryptography allows $n$ parties to share the ability to perform a cryptographic operation (e.g., creating a digital signature(. Any $d$ parties can perform the operation jointly, whereas it is infeasible for at most $d-1$ to do so. In a $(d, n)$-threshold ring signature scheme, the generation of a ring signature for a group of $n$ members requires the involvement of at least $d$ members / signers, and yet the signature reveals nothing about the identities of the signers. 
\newline
\newline
Please notice signers will know each others identity when signing the threshold linkable ring signature. To sign for an event without any acknowledge for any other signers. In this case, we should reduce the problem to use the same pre-defined generate $h$ and create multiple Linkable Ring Signature to the verifier, the verifier can later check any linkability between signatures to count unique signature.
\newline
\newline
We can denote this signature scheme by:
\begin{equation}
SPK\left\lbrace
(\alpha_1,...,\alpha_n) : 
\bigvee_{I \subseteq \mathbb{N}, |I|=d}
\left(
\bigwedge_{i \in I}
{y_i = g_i^{\alpha_i} 
\wedge 
v_i = h_i^{\alpha_i}}
\right) 
\right\rbrace
(M).
\end{equation}

\end{document}
